\chapter{Skills}
\chaplabel{skills}

The success of a wizard is not determined simply by the spells they know and the magics they can perform.
Equally important is their skill at wielding that magic, and even their skill in the more mundane aspects of life.
It's one thing to be able to throw fire, but without the ability to hit what you're aiming for, you're going to be burning down a lot of buildings you didn't mean to.
Similarly, while magic can let you take on someone else's face and voice, doing so is pretty useless if you're a terrible actor.

As such, classes don't simply teach students new techniques; they always contain a practical component to allow students to practise their skills.
Furthermore, schools encourage students to participate in extracurricular activities outside class hours, to help them develop into well-rounded individuals.
All in all, there are many skills, both magical and mundane, that students can develop throughout their education.
These are listed in the following table, while details of each skill are provided in the following sections.

\simpletable{XX}{
	\toprule
	\capital{\mundaneskillbare} & \capital{\magicskillbare}\\
	\midrule
	\capital{\skillref{aim}}\\
	\capital{\skillref{athletics}}\\
	\capital{\skillref{beast-handling}}\\
	\capital{\skillref{charm}}\\
	\capital{\skillrefspeciality{craft}{\variousspeciality}}\\
	\capital{\skillref{fight}}\\
	\capital{\skillref{investigation}}\\
	\capital{\skillref{leadership}}\\
	\capital{\skillrefspeciality{lore}{\variousspeciality}}\\
	\capital{\skillref{observe}}\\
	\capital{\skillrefspeciality{perform}{\variousspeciality}}\\
	\capital{\skillref{rhetoric}}\\
	\capital{\skillref{subterfuge}}\\
	\bottomrule
}

For \skillref{craft}, \skillref{lore}, \skillref{perform}, many different versions of each skill exist.
The skill descriptions below list a number of possible versions of each skill, but a player may select any version they can imagine with the GM's permission.
A student may have a different rank for each version of the skill.
For example, they may have \skillrank{6}{3}{2} for \skillrefspeciality{craft}{woodcarving}, \skillrank{4}{4}{4} for \skillrefspeciality{craft}{glassblowing}, and \skillrank{0}{0}{0} for all other versions of the skill.
Each class or extracurricular that advances these skills will specify a version of the skill, or may allow you to advance any one version of the skill of your choice.

\section{Advancing Skills}
\seclabel{advancing-skills}

There is more to skill than simply level of education; there are several facets in which a student may develop each skill.
Each of a student wizard's skills is represented by a set of three numbers ranging from 0 to 7, such as \skillrank{6}{4}{3}.
The first number represents the depth of their education or training in a skill, and hence determines the maximum they are able to achieve.
The latter two numbers represent the breadth of their education, or how well practised they are.
A student lacking this breadth may have the potential to do very well, but their performance will be less reliable.

All students enter the school with their skills ranked \skillrank{0}{0}{0}.
They still roll 3 dice for {\tests}, so they're not totally incompetent.
But they've all just come out of primary education; they haven't yet specialised.
Characters advance their skills by taking classes and participating in extracurricular activities.
Each class or extracurricular will advance one or more skills.

Each time within a year that a character advances a skill, they add 1 to one of the three numbers representing their skill, working from left to right.
As such, a character cannot benefit from advancing the same skill more than three times in a year.
Furthermore, none of the numbers in a character's skill may exceed their current year level.
At maximum, a character may graduate with a skill ranked \skillrank{7}{7}{7}.

As an example, consider the Athletics skill of Dalton Robins over his first four years of school.
Dalton enters school with his Athletics ranked at \skillrank{0}{0}{0}, the same as everyone else.
In first year he participates in enough sport to advance his Athletics twice, raising the first two numbers of his skill, making it \skillrank{1}{1}{0}.
In second year he does a little less sport and advances Athletics only once, raising only the first number of his skill, making it \skillrank{2}{1}{0}.
In third year he does no sport at all, leaving his skill at \skillrank{2}{1}{0}.
In fourth year, feeling himself growing unfit, Dalton returns to sport with a vengeance and advances Athletics four times.
The first three advances raise all three numbers of his skill, making it \skillrank{3}{2}{1}, but the fourth advance is wasted.

\section{Mundane Skills}
\seclabel{mundane-skills}

Mundane skills are those used in mundane actions---actions that would be possible even for those without magical talent or training.
They are rarely taught in classes at magical schools, and most students will develop them primarily through their extracurricular activities.

\skill{Aim}{aim}

\skill{Athletics}{athletics}

\skill{Beast Handling}{beast-handling}

\skill{Charm}{charm}

\skill{Craft}{craft}

\skill{Fight}{fight}

\skill{Investigation}{investigation}

\skill{Leadership}{leadership}
%TODO: Covers intimidation.

\skill{Lore}{lore}

\skill{Observe}{observe}
%TODO: Covers perception and insight.

\skill{Perform}{perform}

\skill{Rhetoric}{rhetoric}

\skill{Subterfuge}{subterfuge}
%TODO: Covers stealth and deception.

\section{Magic Skills}
\seclabel{magic-skills}

Magical skills represent a student's mastery of the various magical disciplines---they are primarily developed by attending classes.
They are primarily used in the actual practice of magic, though each doubles as a Lore skill concerning magic of the relevant discipline. %TODO: Lore link.
