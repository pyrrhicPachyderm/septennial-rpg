\chapter{Skills \& Tests}
\chaplabel{skills}

The life of a student wizard is filled with tests.
Midterm tests, weekly class assignments, and the dreaded end of semester exams.
But those aren't the sort of tests this chapter is about.

For in truth, a wizard's skills, both magical and mundane, are being tested for more often than they realise.
Whenever they struggle to control a turbulent spell,
whenever they hurl a ball down the field after school,
whenever they try to convince a teacher that a hippogriff really \emph{did} eat their homework.
They may not be receiving a grade for their work, but they are still calling upon the depths of their knowledge and practice.
It is these everyday tests, and the skills they require, that are the subject of this chapter.

\section{Tests}
\seclabel{tests}

When a character attempts an important task at which they may succeed or fail, the GM calls for a {\test}.
The GM should declare which skill the {\test} will use, and the {\targetnumber} ({\tn}) of the {\test}.
The {\targetnumber} is the difficulty of the {\test}---the higher it is, the harder the {\test}.

To make a {\test}, a player looks up their character's relevant skill, as requested by the GM.
This will be a set of three numbers, such as \skillrank{6}{4}{3}.
Roll a number of ten-sided dice (d10s) equal to three plus the first number.
Then, reroll a number of dice up the second number.
Finally, reroll a number of dice up to the third number.
Each of these two rerolls must be done all at once: scoop up all the dice you're going to reroll, and roll them all together.

In a {\test}, you're looking for three things.
Dice that show 7, 8, or 9 are successes.
Dice that show 7 (the magic number), are not just successes, they're critical successes.
Dice that 5 or 6 are only partial successes.
Any other dice, showing 0, 1, 2, 3, or 4, are failures.

To succeed on a {\test}, you need to roll a number of successes (including critical successes) greater than or equal to the {\targetnumber}.
If you don't have enough successes, you may also count partial successes to reach the {\tn}.
However, if you use partial successes in this way, the GM imposes some cost on you for your success.
For example, you might get into a sticky situation, get hurt in the process, leak some excess mana, or the task may take you longer to complete (perhaps making you late to class).
The more partial successes you have to use, the worse the cost should be.
If you don't like the cost, you can always choose to fail the {\test} instead.

Conversely, if you roll critical successes, the GM grants you some additional benefit.
You might improve the situation in a way you hadn't predicted, perform the task more quickly than expected, impress someone who sees your performance, or simply achieve what you're trying more successfully than you'd dared hope.
The more critical successes you roll, the greater the benefit should be.
If you ever roll seven sevens, the feat you perform is likely to become the stuff of legend.

\subsection{An Example Test}

%TODO

\subsection{Target Numbers}
\seclabel{target-numbers}

%TODO

\subsection{Only Players Roll}

%TODO
