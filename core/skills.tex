\chapter{Skills}
\chaplabel{skills}

The success of a wizard is not determined simply by the spells they know and the magics they can perform.
Equally important is their skill at wielding that magic, and even their skill in the more mundane aspects of life.
It's one thing to be able to throw fire, but without the ability to hit what you're aiming at, you're going to be burning down a lot of buildings you didn't mean to.
Similarly, while magic can let you take on someone else's face and voice, doing so is pretty useless if you're a terrible actor.

As such, classes don't simply teach students new techniques; they always contain a practical component to allow students to practise their skills.
Furthermore, schools encourage students to participate in extracurricular activities outside class hours, to help them develop into well-rounded individuals.
All in all, there are many skills, both magical and mundane, that students can develop throughout their education.
These are shown in the \tableref{skill-list} table, while details of each skill are provided in the following sections.

\simpletable{List of Skills}{skill-list}{XX}{
	\toprule
	\capital{\mundaneskillbare} & \capital{\magicskillbare}\\
	\midrule
	\capital{\skillref{aim}} & \capital{\skillref{divination}}\\
	\capital{\skillref{athletics}} & \capital{\skillref{elementalism}}\\
	\capital{\skillref{beast-handling}} & \capital{\skillref{enchanting}}\\
	\capital{\skillref{charm}} & \capital{\skillref{illusion}}\\
	\capital{\skillrefspeciality{craft}{\variousspeciality}} & \capital{\skillref{kinetics}}\\
	\capital{\skillref{fight}} & \capital{\skillref{mentalism}}\\
	\capital{\skillref{investigation}} & \capital{\skillref{metamagic}}\\
	\capital{\skillref{leadership}} & \capital{\skillref{necromancy}}\\
	\capital{\skillrefspeciality{lore}{\variousspeciality}} & \capital{\skillref{restoration}}\\
	\capital{\skillref{observe}} & \capital{\skillref{transmutation}}\\
	\capital{\skillrefspeciality{perform}{\variousspeciality}}\\
	\capital{\skillref{rhetoric}}\\
	\capital{\skillref{subterfuge}}\\
	\bottomrule
}

For \skillref{craft}, \skillref{lore}, and \skillref{perform}, many different versions of each skill exist.
The skill descriptions below list a number of possible versions of each skill, but a player may select any version they can imagine with the GM's permission.
A student may have a different rank for each version of the skill.
For example, they may have 6 ranks in \skillrefspeciality{craft}{woodcarving}, 4 ranks in \skillrefspeciality{craft}{glassblowing}, and 0 ranks in all other versions of the skill.
Each class or extracurricular that advances these skills will specify a version of the skill, or may allow you to advance any one version of the skill of your choice.

\section{Advancing Skills}
\seclabel{advancing-skills}

To advance a skill a student has to practice it, either in class or during their extracurricular activities.
As such, experience (XP) is tracked independently for each skill, and a student must earn XP in a particular skill in order to advance that skill.
All skills begin at rank 0, and their rank increases by 1 for every 4 XP earned in that skill.
The skill's rank is used when making {\tests} with that skill, being added to the result of the dice roll.

Characters can earn XP for skills by receiving advances in those skills.
Each class or extracurricular will advance one or more skills.
However, even the most dedicated study can only improve a skill so far.
As such, there are diminishing returns for advancing a single skill multiple times in the same year.
The first advance earned in a skill each year grants 4 XP.
The second advance earned in the same skill in the same year grants 2 XP.
The third advance in the same year grants only 1 XP.
Fourth and further advances in the same year grant no benefit.

When roleplaying part way through a semester, skill advances from classes and extracurriculars should be considered to be earned halfway through the semester, at the beginning of the mid-semester break.
That is, if playing before the mid-semester break, characters have not yet earned XP from that semester's advances.
If playing during the break, or during the second half of the semester, they have earned that XP, and any corresponding increases in their skill ranks.

\optionalrule{Using Spare XP}{
	If your campaign progresses through the years very slowly, it may take a long time to see the benefit of advancing skills multiple times per year.
	It takes until second year to see any benefit for advancing a skill twice per year, and until third year to see any benefit from advancing it three times per year.
	If you wish to provide some earlier benefit for this, you may allow players to benefit from their spare XP in skills.
	
	Each point of spare XP in a skill (XP earned since last increasing the skill's rank, up to 3) allows a player to take +1 with on a {\test} roll using that skill, once per session.
	This may not be used more than once on the same {\test}, and must be declared before the dice are rolled.
	Doing this does not consume the XP.
}

\section{Mundane Skills}
\seclabel{mundane-skills}

Mundane skills are those used in mundane actions---actions that would be possible even for those without magical talent or training.
They are rarely taught in classes at magical schools, and most students will develop them primarily through their extracurricular activities.

\skill{Aim}{aim}

Used to hit distant targets with something, be it thrown by hand, hit by a bat, or loosed from a bow.
Spells are usually targeted using the appropriate {\magicskill}, but some classes may teach students to aim them using their mundane ability. %TODO: Write this class and provide the relevant link.

\skill{Athletics}{athletics}

Used to run, jump, climb, swim, hike, ski, and generally to get about the place.
While many sports will require skill in \skillref{aim} or \skillref{fight}, most sports depend heavily, even primarily, on \skillref{athletics}.

\skill{Beast Handling}{beast-handling}

Used to understand beasts and interact with them: to calm them, tame them, ride them, train them, and command them.
While normally applied to animals, magical or mundane, some magical plants and fungi have sufficient will and motility that this skill can be useful in dealing with them.

\skill{Charm}{charm}

Used to make people like you: to befriend them, to build rapport with them, or to disarm them with a smile.
While \skillref{rhetoric} is used to make a persuasive argument, people who like you enough will often take your suggestions without any argument required.

\skill{Craft}{craft}

Used to make things with your hands.
Each \skillref{craft} skill is learned separately, and represents the ability to practise a particular craft.
Available crafts include the following:
\begin{itemize}
	\item Blacksmithing
	\item Carpentry
	\item Cooking
	\item Glassblowing
	\item Jewellery
	\item Masonry
	\item Pottery
	\item Seamstressing
	\item Woodcarving
	%TODO: Evaluate and maybe expand.
\end{itemize}

\skill{Fight}{fight}

Used to engage in hand-to-hand combat, either with weapons or without.
This involves both harming your opponent and preventing them from harming you.

\skill{Investigation}{investigation}

Used to uncover things by extended investigative effort.
This includes rifling a desk drawer for a document, ransacking a room for equipment, searching a library for pertinent information, or even unearthing a piece of juicy gossip.

\skill{Leadership}{leadership}

Used to command obedience from a position of authority or superiority.
This includes leading subordinates, whether they follow you willingly or due to your institutionally invested authority.
It also includes intimidating those who are physically or socially inferior to you.

\skill{Lore}{lore}

Used to know and recall information.
Each \skillref{lore} skill is learned separately, and represents knowledge of a different field.
Available fields of knowledge include the following (knowledge of magic is covered by {\magicskills}):
\begin{itemize}
	\item Botany
	\item Geography
	\item History
	\item Medicine
	\item Mycology %TODO: Roll this into Botany?
	\item Philosophy
	\item Politics
	\item Religion
	\item Zoology
	%TODO: Evaluate and maybe expand.
\end{itemize}

\skill{Observe}{observe}

Used to notice things, by vision, hearing, or even smell.
This includes spotting something out of place, such as spotting a piece of lab equipment out of place, or hearing someone sneaking up on you.
It also includes watching people carefully, such that you might pick up on signs they're hiding something or lying.

\skill{Perform}{perform}

Used to entertain people by putting on a show.
Each \skillref{lore} skill is learned separately, and represents a different form of entertainment.
Available methods of performance include the following:
\begin{itemize}
	\item Acting
	\item Dancing
	\item Drums
	\item Flute
	\item Singing
	\item Trumpet
	\item Violin
	%TODO: Evaluate and maybe expand.
\end{itemize}

\skill{Rhetoric}{rhetoric}

Used to persuade people with reason, logic, debate, and eloquence.
While \skillref{charm} can make someone like you, \skillref{rhetoric} can make them agree with you, even despite how they might feel about you personally.
\capital{\skillref{rhetoric}} is also used for most writing, a useful skill for a student.

\skill{Subterfuge}{subterfuge}

Used to conceal your covert activities.
This includes sneaking, hiding, stealing, smuggling, and even lying.

\section{Magic Skills}
\seclabel{magic-skills}

Magical skills represent a student's mastery of the various magical disciplines---they are primarily developed by attending classes.
They are primarily used in the actual practice of magic, though each doubles as a \skillref{lore} skill concerning magic of the relevant discipline.

\skill{Divination}{divination}

Used to obtain information by magic.
\capital{\skillref{divination}} allows a wizard to scry on remote locations, sense things beyond normal perception, pierce illusions, and otherwise to discover whatever they cannot learn mundanely.

\skill{Elementalism}{elementalism}

Used to manipulate the four classical elements: water, earth, fire, and air.
While not as broad in target as \skillref{kinetics}, \skillref{elementalism} is far more versatile in what it can do with the elements it affects.

\skill{Enchanting}{enchanting}

Used to bind magical power into objects.
\capital{\skillref{enchanting}} can invest objects with the ability to generate magical effects independently of a wizard, among other things.
It is also used to create casting tools and potions.

\skill{Illusion}{illusion}

Used to fool the senses by tricks of light, sound, scent, or other phenomena.
\capital{\skillref{illusion}} can efficiently create large or intricate effects, with the caveat that they have no physical substance.
Used to deceive or mislead, however, they can be very effective.

\skill{Kinetics}{kinetics}

Used to manipulate motion and forces.
In its most obvious form, this allows telekinesis.
However, it can also be applied to allow flight, warding, teleportation, and more.

\skill{Mentalism}{mentalism}

Used to manipulate the minds of people and beasts.
This includes putting them to sleep, subtly manipulating their thoughts, altering their memories, or even controlling their minds outright.
Applied to people, this is widely considered to be one of the more ethically troublesome forms of magic.

\skill{Metamagic}{metamagic}

Used to manipulate magic itself.
\capital{\skillref{metamagic}} spells often affect other spells, empowering them, cancelling them, or otherwise altering them.
They may also affect {\mana} itself, generating it, or funnelling more of it into a single effect.

\skill{Necromancy}{necromancy}

Used to manipulate the dead, or to bring about death in the living.
In the former application, \skillref{necromancy} allows the animation of undead or communion with spirits.
In the latter application, it directly weakens, wounds, or kills, without any need for the fire of \skillref{elementalism} or the bludgeon of \skillref{kinetics}.
\capital{\skillref{necromancy}} is perhaps the most feared form of magic, and its use is often highly regulated.

\skill{Restoration}{restoration}

Used to repair damage, restoring things to their original state.
\capital{\skillref{restoration}} is most well known for its use upon people: healing.
However, it can also be used to heal animals and plants, and even to repair damaged or destroyed objects.

\skill{Transmutation}{transmutation}

Used to transform things into other things.
This may affect nonliving things, such as turning lead into gold; living things, such as turning people into animals; or turn one into the other, such as turning a person to stone.
