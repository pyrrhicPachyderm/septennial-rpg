\chapter{Fundamentals of Magic}
\chaplabel{magic-rules}

Learning magic isn't easy.
Student wizards would need to spend seven years at school if it was.
Casting spells is an exercise in precise gestures, carefully recited incantations, and intense mental focus.
Enchanting is arguably even more complicated.
This chapter, therefore, shall present the cliff notes on the matter. %NB: "Cliff notes" is a genericised trademark; this is acceptable.
That is, the rules for the use of magic in game.

\section{Currencies of Magic}

There are two essential ``currencies'' of magic, used to determine the magnitude of a magical effect that a wizard can produce: {\focus} and {\mana}.
\capital{\focus} represents the basic spellcasting ability a wizard always has available, and also their ability to concentrate upon ongoing spell effects.
\capital{\mana} is a pool of excess mental energy that a wizard can utilise to push themselves and produce more powerful effects than their basic {\focus} would allow.
A third statistic, {\overchannel}, determines how much {\mana} a wizard can utilise on a given spell.
\capital{\focus}, {\mana}, and {\overchannel} are all so essential to a wizard that they are not taught in specific classes, but generally trained in all practical activities students perform.
As such, they advance automatically with a student's current year level, as shown in the \tableref{magic-currencies} table.

\simpletable{Magical Power by Year}{magic-currencies}{rrrr}{
	\toprule
	Year & \capital{\focus} & \capital{\overchannel} & \capital{\mana}\\
	\midrule
	1 & 3 & 2 & 6\\
	2 & 4 & 2 & 8\\
	3 & 4 & 3 & 10\\
	4 & 5 & 3 & 12\\
	5 & 5 & 4 & 14\\
	6 & 6 & 4 & 16\\
	7 & 6 & 5 & 18\\
	\bottomrule
}

\subsection{Focus}
\seclabel{focus}

Casting spells requires intense mental focus.
Each spell has a {\focus} cost, which may be modified by various factors.
In order to cast a spell, a student must be able to muster the requisite {\focus}.
\capital{\focus} is not consumed by casting a spell---it is simply a threshold that must be met.
A student's base {\focus}, as given in the \tableref{magic-currencies} table, thus determines the maximum difficulty of spell they can cast without extra effort or other aid.

\subsubsection{Concentration}
\seclabel{concentration}

Some spells with ongoing effects take ongoing concentration to maintain.
After a student completes the casting of such a spell their {\focus} is reduced by the {\concentration} cost of the spell, for as long as they wish to maintain the effect.
A student may cease {\concentrating} on any spell at any time, and it immediately ends if they do so.
They also cease {\concentrating} if they go to sleep or otherwise fall unconscious.

A student may {\concentrate} on multiple spells simultaneously, and their {\concentration} costs stack to further reduce their {\focus}.
However, a student may never {\concentrate} on so much as to reduce their base {\focus} below 0.
They must cease {\concentrating} on one of their spells to recover some {\focus} if this would occur.

\subsection{Mana}
\seclabel{mana}

\capital{\mana} is a kind of magical mental energy that wizards can expend to power their magic, allowing more powerful effects than their {\focus} allows them to readily produce.
The \tableref{magic-currencies} table shows a student's maximum {\mana}, which they may expend points from.
They may recover all their expended {\mana} with an hour of rest and relaxation.
This cannot be done during classes, but the lunch break affords sufficient rest as long as students refrain from doing anything too taxing.
A student cannot rest to recover while {\mana} while {\concentrating} on any spells.

\subsubsection{Overchannelling}
\seclabel{overchannel}

The primary use of {\mana} is provide additional power to spells that a student cannot cast with their base {\focus} alone.
Whenever casting a spell, a student may spend an amount {\mana} up to their {\overchannel} in order to increase their {\focus} for that spell by the {\mana} spent.
The increased {\focus} only benefits the initial casting of the spell---it has no effect upon {\concentration}.

\subsubsection{Bound Mana}
\seclabel{bound-mana}

%TODO: Make "enchanting" link appropriately, to chapter rather than skill.
\capital{\mana} is also essential in enchanting.
Enchanting is the art of binding a portion of one's {\mana} into a physical object.
Binding {\mana} firstly requires you to expend the requisite amount of {\mana} as you enchant the object.
While the object remains enchanted, your maximum {\mana} is reduced by the amount you bound into the enchantment.
This reduces the amount you restore when you rest, as well as the maximum amount you may hold when restoring {\mana} by any other means.

When you rest and recover {\mana}, you may reclaim {\mana} from any number of objects you have enchanted, ending the enchantment on that item.
The enchantment ends, but your maximum and current {\mana} are restored.
If an enchanted object has been destroyed or disenchanted, you automatically recover its {\mana} when you next rest.

\section{Spells}
\seclabel{spells}

Most magic is performed by casting spells.
Each spell a wizard knows is a particular series of incantations, gestures, and mental exercises that harness magical energies to produce a particular effect.
Spells are normally learned in classes, though a few minor ones may already be known by students when they arrive at school.

Most spells have a similar set of requirements for casting them, although certain spells will specify that they vary in these requirements.
Firstly, a student must have sufficient {\focus}, by {\overchannelling} if necessary.
Secondly, most spells require the caster to speak an incantation and perform the required gestures, to shape the arcane energies being harnessed.
As such, the student must be able to speak and make sound, and must have one unrestrained hand which is either empty, or only holding a {\castingtool}.
%TODO: Silent spells and still spells waive these requirements.
Thirdly, most spells require a few seconds to cast.
As such, most spells are too slow to be cast in reaction to immediate danger, such as falling from a tower or having a fireball flying at you.
%TODO: Reactive spells and extended spells alter this requirement.

In most instances, simply casting spells does not require a {\test}.
Characters are assumed to be able to cast any spell taught in a class they have passed, given sufficient {\focus}.
\capital{\tests} are only called for when it is uncertain that a spell will produce the desired effect, most commonly when another character is opposing it.
For example, a {\test} may be required to hit a moving target with a fireball, or to make an illusion sufficiently convincing to fool someone.

One exception to this rule is when a character attempts to use magic from a class they are taking, but have not yet finished.
In this case, the GM may call for a {\test} to perform the magic at all, with the {\tn} influenced by how far the student is through the semester.
