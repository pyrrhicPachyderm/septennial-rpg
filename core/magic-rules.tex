\chapter{Fundamentals of Magic}
\chaplabel{magic-rules}

Learning magic isn't easy.
Student wizards would need to spend seven years at school if it was.
Casting spells is an exercise in precise gestures, carefully recited incantations, and intense mental focus.
Enchanting is arguably even more complicated.
This chapter, therefore, shall present the cliff notes on the matter. %NB: "Cliff notes" is a genericised trademark; this is acceptable.
That is, the rules for the use of magic in game.

\section{Currencies of Magic}

There are two essentially ``currencies'' of magic, used to determine the magnitude of a magical effect that a wizard can produce: {\focus} and {\mana}.
\capital{\focus} represents the basic spellcasting ability a wizard always has available, and also their ability to concentrate upon ongoing spell effects.
\capital{\mana} is a pool of excess mental energy that a wizard can utilise to push themselves and produce more powerful effects than their basic {\focus} would allow.
A third statistic, {\overchannel}, determines how much {\mana} a wizard can utilise on a given spell.
\capital{\focus}, {\mana}, and {\overchannel} are all so essential to a wizard that they are not taught in specific classes, but generally trained in all practical activities students perform.
As such, they advance automatically with a student's current year level, as shown in the following table:
\simpletable{rrrr}{
	\toprule
	Year & \capital{\focus} & \capital{\overchannel} & \capital{\mana}\\
	\midrule
	1 & 3 & 2 & 6\\
	2 & 4 & 2 & 8\\
	3 & 4 & 3 & 10\\
	4 & 5 & 3 & 12\\
	5 & 5 & 4 & 14\\
	6 & 6 & 4 & 16\\
	7 & 6 & 5 & 18\\
	\bottomrule
}
