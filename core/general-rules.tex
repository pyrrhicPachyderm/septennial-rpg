\chapter{Tests \& Tribulations}
\chaplabel{general-rules}

%TODO: Rewrite this introduction to address that this is more of a general rules chapter than specifically tests.

The life of a student wizard is filled with tests.
Midterm tests, weekly class assignments, and the dreaded end of semester exams.
But those aren't the sort of tests this chapter is about.

For in truth, a wizard's skills, both magical and mundane, are being tested for more often than they realise.
Whenever they struggle to control a turbulent spell,
whenever they hurl a ball down the field after school,
whenever they try to convince a teacher that a hippogriff really \emph{did} eat their homework.
They may not be receiving a grade for their work, but they are still calling upon the depths of their knowledge and practice.
It is these everyday tests that are the subject of this chapter.

\section{Tests}
\seclabel{tests}

When a character attempts an important task at which they may succeed or fail, the GM calls for a {\test}.
The GM should declare which skill the {\test} will use, and the {\targetnumber} ({\tn}) of the {\test}.
The {\targetnumber} is the difficulty of the {\test}---the higher it is, the harder the {\test}.

To make a {\test}, a player looks up their character's relevant skill, as requested by the GM.
This will be a set of three numbers, such as \skillrank{6}{4}{3}.
Roll a number of ten-sided dice (d10s) equal to three plus the first number.
Then, reroll a number of dice up the second number.
Finally, reroll a number of dice up to the third number.
Each of these two rerolls must be done all at once: scoop up all the dice you're going to reroll, and roll them all together.

In a {\test}, you're looking for three things.
Dice that show 7, 8, or 9 are successes.
Dice that show 7 (the magic number), are not just successes, they're critical successes.
Dice that 5 or 6 are only partial successes.
Any other dice, showing 0, 1, 2, 3, or 4, are failures.

To succeed on a {\test}, you need to roll a number of successes (including critical successes) greater than or equal to the {\targetnumber}.
If you don't have enough successes, you may also count partial successes to reach the {\tn}.
However, if you use partial successes in this way, the GM imposes some cost on you for your success.
For example, you might get into a sticky situation, get hurt in the process, leak some excess {\mana}, or the task may take you longer to complete (perhaps making you late to class).
The more partial successes you have to use, the worse the cost should be.
If you don't like the cost, you can always choose to fail the {\test} instead.

Conversely, if you roll critical successes, the GM grants you some additional benefit.
You might improve the situation in a way you hadn't predicted, perform the task more quickly than expected, impress someone who sees your performance, or simply achieve what you're trying more successfully than you'd dared hope.
The more critical successes you roll, the greater the benefit should be.
If you ever roll seven sevens, the feat you perform is likely to become the stuff of legend.

\subsection{An Example Test}

As an example, suppose Bertram Harrison is scaling the observatory tower to get into the divination lab after dark.
Climbing up the outside of a building is risky business, sot he GM calls for an \skillref{athletics} {\test}.
Scaling a sheer stone wall in the dark would normally be very difficult, but there's a drainpipe to help him.
It's still pretty difficult, though, so the GM assigns a {\tn} of 4.
Bertram's a third year student, and he's participated in quite a lot of sport over the past few years, so he has an \skillref{athletics} skill of \skillrank{3}{3}{2}.

Bertram's player rolls the standard three dice, plus an additional three for the first number of Bertram's \skillref{athletics} score.
He rolls 9, 6, 5, 4, 3, and 0: one success, two partial successes, and three failures.
Even using his partial successes, that's not enough to meet the {\tn} of 4, but he now gets to reroll some dice.
Using the second number of Bertram's \skillref{athletics} score, he rerolls the three failures.
They come up 5, 5, and 1: two more partial successes and still one failure.
He now has one success and four partial successes.
That's enough to meet the {\tn} of 4, but using three partial successes.
He can do better.
Using the third number of Bertram's \skillref{athletics}, he can reroll two more dice.
He rerolls the failure, and one of the partial successes---even if these both roll failures, he has enough partial successes left to pass the {\test}.
These two dice come up 7 and 2: still one failure, but one critical success!

The dice now show 9, 7, 6, 5, 5, and 2.
Bertram's player now has two successes and three partial successes.
He decides to use two of the partial successes to meet the {\tn} of 4 and pass the {\test}.
He also notes that one of his successes as critical.

Bertram passed the {\test}, so he successfully scales the tower and climbs into the divination lab through an open window.
However, he used two partial successes, so the GM imposes a substantial penalty: he dislodges a stone halfway up the tower, which crashes loudly to the ground.
Someone will surely have heard that, and curfew patrol will be searching the tower soon.
He also rolled one critical success, so the GM gives him a bonus: he climbs the tower very quickly, and he's in through the window within seconds of dislodging the stone.
That should buy him at least a few minutes to search the room before anyone arrives to investigate.
Bertram breathes a sigh of relief that at least he didn't fall himself, then begins ransacking the lab supplies.

\subsection{Target Numbers}
\seclabel{target-numbers}

A {\targetnumber} ({\tn}) represents the difficulty of a {\test}.
It is a number from 1 to 10, assigned by the GM based on the situation.
While other RPGs might apply bonuses or penalties to a player's rolls, in \titleemph{Septennial}, the difficulty of a {\test} is always adjusted by changing by the {\tn}.
The table below summarises the difficulty that each possible {\tn} represents: consistency describes the ability to normally pass a {\test} without relying on partial successes, while struggling describes accepting partial successes.
%NB: As a guideline, a "skilled X year" refers to X,X,floor(X/2), and an "expert graduate" refers to 7,7,7.

\simpletable{rX}{
	\toprule
	{\tn} & Difficulty\\
	\midrule
	%TODO: Add one-word summaries of the difficulties?
	1 & Even an unskilled student will manage this most of the time.\\
	2 & An unskilled student can manage this half the time; one with a little skill can usually manage it.\\
	3 & A skilled third year can achieve this consistently; even a skilled first year will struggle.\\
	4 & A skilled fourth year can achieve this consistently; a skilled second year can usually struggle through.\\
	5 & A skilled sixth year can achieve this consistently; a skilled third year can usually struggle through.\\
	6 & A skilled seventh year can achieve this consistently; a skilled fourth year can usually struggle through.\\
	7 & Even an expert graduate isn't quite consistent, but even a skilled fifth year can usually struggle through.\\
	8 & Even an expert graduate usually struggles a little, but even a skilled sixth year can usually struggle through.\\
	9 & Even an expert graduate will struggle considerably to achieve this.\\
	10 & Even an expert graduate will normally fail this outright.\\
	\bottomrule
}

\subsection{Using Tests}

Only the player characters are ever called to make {\tests}.
When a character run by the GM acts against a player character, the player character can be called to make a {\test} to resist or avoid the action.
The skill used depends on how they try to resist.
For example, if a hippogriff swings its talons at them, they might {\test} \skillref{fight} to block it with their staff, {\test} \skillref{athletics} to leap out of the way, or {\test} an appropriate {\magicskill} to cast a defensive spell in reaction.

Be careful not to call for a {\test} when it's not necessary.
If an action is a simple one that the character should be able to routinely perform, such as walking through a door or ransacking a room for something that isn't hidden, it doesn't require a {\test}.
If a character attempts something impossible, such as jumping over the moon or fast-talking the headmaster into giving up his position to a first year student, they fail without a {\test} being rolled.
Lastly, if there is no penalty for failure, there is no need for a {\test}.
If the character will keep on trying until they succeed, there's no need to make the player keep rolling {\tests}.

Note that partial failures and critical successes will occur quite frequently, especially as players begin rolling more dice.
So they don't need to be hugely impactful, but they should make some difference.
Players are free to suggest costs or benefits resulting from partial or critical successes, though the GM always has final say.
If struggling to find both a positive and negative effect to suit both critical and partial successes in the same roll, the GM can allow critical and partial successes to cancel each other, one for one.
