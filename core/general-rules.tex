\chapter{Tests \& Tribulations}
\chaplabel{general-rules}

%TODO: Rewrite this introduction to address that this is more of a general rules chapter than specifically tests.

The life of a student wizard is filled with tests.
Midterm tests, weekly class assignments, and the dreaded end of semester exams.
But those aren't the sort of tests this chapter is about.

For in truth, a wizard's skills, both magical and mundane, are being tested for more often than they realise.
Whenever they struggle to control a turbulent spell,
whenever they hurl a ball down the field after school,
whenever they try to convince a teacher that a hippogriff really \emph{did} eat their homework.
They may not be receiving a grade for their work, but they are still calling upon the depths of their knowledge and practice.
It is these everyday tests that are the subject of this chapter.

\section{Tests}
\seclabel{tests}

When a character attempts an important task at which they may succeed or fail, the GM calls for a {\test}.
The GM should declare which skill the {\test} will use, and the {\targetnumber} ({\tn}) of the {\test}.
The {\targetnumber} is the difficulty of the {\test}---the higher it is, the harder the {\test}.

To make a {\test}, roll two dice, as shown in the \tableref{test-dice} table, and add them together with your relevant skill.
If your total is greater than or equal to the {\tn}, you succeed.
You manage what you were trying to do, and get what you want.
Otherwise, you fail.
Your attempt was unsuccessful, and you suffer the consequences.

Students are young and still coming into their prime, so, regardless of their skills, their abilities are constantly improving.
As such, the dice you roll for a {\test} are dependent upon your year level, and improve over time, according to the \tableref{test-dice} table.

\simpletable{Test Dice by Year}{test-dice}{rrrr}{
	\toprule
	Year & Dice & Average\\
	\midrule
	1 & 2d6 & 7\\
	2 & 1d6+1d8 & 8\\
	3 & 2d8 & 9\\
	4 & 1d8+1d10 & 10\\
	5 & 2d10 & 11\\
	6 & 1d10+1d12 & 12\\
	7 & 2d12 & 13\\
	\bottomrule
}

\subsection{Target Numbers}
\seclabel{target-numbers}

A {\targetnumber} ({\tn}) represents the difficulty of a {\test}.
It is assigned by the GM based on the nature of the situation and the feasibility of the task the character is attempting.
While other RPGs might apply bonuses or penalties to a player's rolls, in \titleemph{Septennial}, the difficulty of a {\test} is always adjusted by changing by the {\tn}.

Much as a student becomes generally more competent as they progress through their schooling, the magnitude of the problems they face will also increase.
As such, the {\tns} of the tasks a student is attempting should also increase.
In truth, they may also still face many of the same challenges as they did in previous years, but those tasks have now become routine for them and thus unworthy of narritive attention.

The \tableref{test-dice} table provides the average result of the dice rolled for a {\test} at each year level, which may be of use to the GM in determining an appropriate {\tn}.
An unskilled student of the appropriate year level will succeed on a {\test} with this {\tn} just slightly more than half the time.
A skilled student has a greater chance of success, and the rank of a student's skills will also improve over time.
As such, a skilled student has a good chance of success at more difficult tasks, particularly in higher year levels.
Remember, however, that there will always be many skills that each student does not learn.
As such, students can still be challenged by {\tns} similar to those presented in the table, and these will simply give skilled students an opportunity to shine.

\subsection{Using Tests}

Only the player characters are ever called to make {\tests}.
When a character run by the GM acts against a player character, the player character can be called to make a {\test} to resist or avoid the action.
The skill used depends on how they try to resist.
For example, if a hippogriff swings its talons at them, they might {\test} \skillref{fight} to block it with their staff, {\test} \skillref{athletics} to leap out of the way, or {\test} an appropriate {\magicskill} to cast a defensive spell in reaction.

Be careful not to call for a {\test} when it's not necessary.
If an action is a simple one that the character should be able to routinely perform, such as walking through a door or ransacking a room for something that isn't hidden, it doesn't require a {\test}.
If a character attempts something impossible, such as jumping over the moon or fast-talking the headmaster into giving up his position to a first year student, they fail without a {\test} being rolled.
Lastly, if there is no penalty for failure, there is no need for a {\test}.
If the character will keep on trying until they succeed, there's no need to make the player keep rolling {\tests}.
