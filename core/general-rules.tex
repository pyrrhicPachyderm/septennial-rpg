\chapter{Tests \& Tribulations}
\chaplabel{general-rules}

%TODO: Rewrite this introduction to address that this is more of a general rules chapter than specifically tests.

The life of a student wizard is filled with tests.
Midterm tests, weekly class assignments, and the dreaded end of semester exams.
But those aren't the sort of tests this chapter is about.

For in truth, a wizard's skills, both magical and mundane, are being tested for more often than they realise.
Whenever they struggle to control a turbulent spell,
whenever they hurl a ball down the field after school,
whenever they try to convince a teacher that a hippogriff really \emph{did} eat their homework.
They may not be receiving a grade for their work, but they are still calling upon the depths of their knowledge and practice.
It is these everyday tests that are the subject of this chapter.

\section{Tests}
\seclabel{tests}

When a character attempts an important task at which they may succeed or fail, the GM calls for a {\test}.
The GM should declare which skill the {\test} will use, and the {\targetnumber} ({\tn}) of the {\test}.
The {\targetnumber} is the difficulty of the {\test}---the higher it is, the harder the {\test}.

To make a {\test}, roll two dice, as shown in the \tableref{test-dice} table, and add them together with your relevant skill.
If your total is greater than or equal to the {\tn}, you succeed.
You manage what you were trying to do, and get what you want.
Otherwise, you fail.
Your attempt was unsuccessful, and you suffer the consequences.

Students are young and still coming into their prime, so, regardless of their skills, their abilities are constantly improving.
As such, the dice you roll for a {\test} are dependent upon your year level, and improve over time, according to the \tableref{test-dice} table.

\simpletable{Test Dice by Year}{test-dice}{rrrr}{
	\toprule
	Year & Dice & Average\\
	\midrule
	1 & 2d6 & 7\\
	2 & 1d6+1d8 & 8\\
	3 & 2d8 & 9\\
	4 & 1d8+1d10 & 10\\
	5 & 2d10 & 11\\
	6 & 1d10+1d12 & 12\\
	7 & 2d12 & 13\\
	\bottomrule
}

\subsection{Target Numbers}
\seclabel{target-numbers}

A {\targetnumber} ({\tn}) represents the difficulty of a {\test}.
It is assigned by the GM based on the nature of the situation and the feasibility of the task the character is attempting.

Much as a student becomes generally more competent as they progress through their schooling, the magnitude of the problems they face will also increase.
As such, the {\tns} of the tasks a student is attempting should also increase.
In truth, they may also still face many of the same challenges as they did in previous years, but those tasks have now become routine for them and thus unworthy of narrative attention.

The \tableref{test-dice} table provides the average result of the dice rolled for a {\test} at each year level, which may be of use to the GM in determining an appropriate {\tn}.
An unskilled student of the appropriate year level will succeed on a {\test} with this {\tn} just slightly more than half the time.
A skilled student has a greater chance of success, and the rank of a student's skills will also improve over time.
As such, a skilled student has a good chance of success at more difficult tasks, particularly in higher year levels.
Remember, however, that there will always be many skills that each student does not learn.
As such, students can still be challenged by {\tns} similar to those presented in the table, and these will simply give skilled students an opportunity to shine.

\subsection{Using Tests}

Only the player characters are ever called to make {\tests}.
When a character run by the GM acts against a player character, the player character can be called to make a {\test} to resist or avoid the action.
The skill used depends on how they try to resist.
For example, if a hippogriff swings its talons at them, they might {\test} \skillref{fight} to block it with their staff, {\test} \skillref{athletics} to leap out of the way, or {\test} an appropriate {\magicskill} to cast a defensive spell in reaction.

Be careful not to call for a {\test} when it's not necessary.
If an action is a simple one that the character should be able to routinely perform, such as walking through a door or ransacking a room for something that isn't hidden, it doesn't require a {\test}.
If a character attempts something impossible, such as jumping over the moon or fast-talking the headmaster into giving up his position to a first year student, they fail without a {\test} being rolled.
Lastly, if there is no penalty for failure, there is no need for a {\test}.
If the character will keep on trying until they succeed, there's no need to make the player keep rolling {\tests}.

\section{Injury \& Healing}
\seclabel{injury}

Though safer than many places, school is never a totally safe environment.
Magical mishaps, accidents on the sports field, brawls between students, an invasion by a dark wizard---many things can lead to students being injured.

\subsection{Damage}
\seclabel{damage}

When something happens that would injure someone, it does some amount of {\damage}, determined by rolling some dice.
Unless {\tests}, even non-player characters may roll for {\damage}.
Spells, weapons, and attacks by mundane or magical means each specify how much {\damage} they deal.
In other cases, the GM simply chooses an appropriate number of dice to roll for {\damage}.

\subsubsection{Toughness}
\seclabel{toughness}

All characters and creatures have a {\toughness}, which determines how well they can withstand {\damage}.
Creatures with a higher {\toughness} suffer less severe {\wounds} from the same amount of {\damage}, and can withstand greater amounts of {\damage} without dying.
By default, all humans, including all player characters, have a {\toughness} of 4, though this may be modified by {\virtues} or {\flaws}.
Non-humans can have a much wider range of values for {\toughness}.

\subsection{Wounds}
\seclabel{wounds}

Injury, to player or non-player characters, is represented in the form of {\wounds}.
Each instance of one or more {\damage} inflicts a {\wound} upon the target.
\capital{\wounds} have a severity---light, moderate, severe, incapacitating, or lethal---and a description.
The severity is determined by comparing the {\damage} dealt to the target's {\toughness}, while the description is provided by the GM based on what seems narratively appropriate.
The description should reflect the severity, and should specify at least where on the target's body the {\wound} has been dealt.

A character may only have a limited number of {\wounds} of each severity: two each of light, moderate, and severe {\wounds}, and one incapacitating {\wound}.
If they would take a {\wound} in excess of this at any severity, they suffer a more severe {\wound} instead.
A character with any fatal {\wounds} is dead.

\subsubsection{Inflicting Wounds}

To determine the severity of a {\wound}, compare the {\damage} dealt to target's {\toughness}.
\capital{\damage} up to the target's {\toughness} deals a light {\wound}.
\capital{\damage} exceeding this, but not exceeding twice the target's {\toughness}, deals a moderate {\wound}.
Up to three times the target's {\toughness} deals a severe {\wound}, and up to four times the target's {\toughness} deals an incapacitating {\wound}.
\capital{\damage} exceeding four times the target's {\toughness} toughness is fatal.
The \tableref{wound-ranges} table provides a handy reference for some common values of {\toughness}.

\simpletable{Wound Ranges by Toughness}{wound-ranges}{rccccc}{
	\toprule
	\rothead{\capital{\toughness}} & \rothead{Light} & \rothead{Moderate} & \rothead{Severe} & \rothead{Incapacitating} & \rothead{Fatal}\\
	\midrule
	0 & --- & --- & --- & --- & 1+\\
	1 & 1 & 2 & 3 & 4 & 5+\\
	2 & 1--2 & 3--4 & 5--6 & 7--8 & 9+\\
	3 & 1--3 & 4--6 & 7--9 & 10--12 & 13+\\
	4 & 1--4 & 5--8 & 9--12 & 13--16 & 17+\\
	5 & 1--5 & 6--10 & 11--15 & 16--20 & 21+\\
	\bottomrule
}

If {\damage} would inflict a {\wound} of a certain severity, but the character already has their full allotment of {\wounds} at that severity, they take a {\wound} of their next available greater severity instead.
This may cause a fatal {\wound} if a character already has an incapacitating {\wound}.

The description of a {\wound}, including where on the body it is inflicted, is ultimately left at the discretion of the GM, as influenced by the severity and any narrative factors.
However, if a player wishes to {\wound} a particular part of an opponent's body, they may declare their intention to the GM to do so.
The GM may simply allow them to do so, call for a {\test} to do so, or make an existing {\test} to hit the target more difficult, due to the complication in aiming for a particular location.
Again, the narrative should inform this choice: it's easier to hit a particular arm with a sword than it is with a fireball hurled from the opposite end of a field.

\subsubsection{Wound Penalties}

Any character who suffers a fatal {\wound} immediately dies.
Characters who suffer an incapacitating {\wound} are rendered unconscious.
They cannot be roused for at least a few minutes, unless the {\wound} is healed, and even when roused, cannot achieve anything without assistance as long as they still have an incapacitating {\wound}.
Lesser {\wounds}---light, moderate, and severe---carry two penalties for player characters.

\simpletable{Wound Penalties}{wound-penalties}{lrr}{
	\toprule
	Severity & \capital{\focus} Penalty & \capital{\test} Penalty\\
	\midrule
	Light & $-1$ & $-3$\\
	Moderate & $-2$ & $-6$\\
	Severe & $-3$ & $-9$\\
	\bottomrule
}

Firstly, {\wounds} apply a penalty to a character's {\focus}, as the pain makes casting spells more difficult.
This may require a character to lose {\concentration} on some spells when they are {\wounded}
It is, however, possible to push through this pain with increased mental effort: a character's {\overchannel} is increased by the same amount as any {\focus} lost due to {\wounds}.
Ignoring or reducing the {\focus} penalty for any reason also reduces the amount of {\overchannel} gained.

Secondly, {\wounds} apply a penalty to all {\tests} that their description would indicate they should penalise (as adjudicated by the GM).
For example, a {\wounded} leg would penalise {\tests} made to run, and a {\wounded} arm would penalise {\tests} made to use a sword.
This will usually only penalise physical {\tests}, not mental ones.
It should not penalise {\tests} to use magic, as that is covered by the {\focus} penalty.

The penalties caused by {\wounds} of each severity are shown in the \tableref{wound-penalties} table.
\capital{\wound} penalties never stack: only the greatest relevant {\wound} penalty ever applies.
For the {\focus} penalty, this will always be based on the character's most severe {\wound}.
For the {\test} penalty, this may be based on a less severe {\wound}, if the more severe {\wound} would not impede the relevant action.

Non-player characters can be {\wounded} in the same way as player characters.
As non-player characters can never make {\tests}, the penalty from a non-player character's {\wounds} is instead applied as a bonus to a player character's {\tests}, when such {\tests} might benefit from the {\wound}.
For example, a {\test} made to chase after a creature with a {\wounded} leg, or to sword fight against a person with a {\wounded} arm, would gain a bonus from the most severe relevant {\wound}.
GMs are not required to track {\focus} for non-player spellcasters, so this penalty is not applied numerically, but GMs are nonetheless encouraged to bear the effect of {\wounds} in mind.

\subsection{Healing \& Recovery}

Recovering naturally from {\wounds} is a slow process.
It requires some period of time, after which the wound is reduced the next-lowest severity (and likely has its description adjusted to match).
The recovery period is given in the \tableref{wound-recovery} table.
For more severe wounds, natural healing will also require first aid and some care not exacerbate the wound during the healing period: a black eye will be little problem, but a broken leg will require splinting and the use of a crutch for a while.

\simpletable{Wound Recovery Time}{wound-recovery}{ll}{
	\toprule
	Severity & Recovery Time\\
	\midrule
	Light & 1 day\\
	Moderate & 1 week\\
	Severe & 1 month\\
	Incapacitating & 1 month\\
	\bottomrule
}

Thankfully, the ready availability of magic means that students are rarely required to recover from any but light {\wounds} naturally.
The school nurse can always heal even incapacitating {\wounds} within a day.
However, while the school nurse will never decline a student treatment, their ministrations often come with a number of questions about how a student's {\wounds} were sustained.
Students wishing to avoid their activities being scrutinised might, therefore, wish to avoid the nurse.
Healing can instead be sought from students studying restoration, but this can come with its own problems: a substandard quality of healing, or a price demanded in exchange. %TODO: Make "restoration" link.
