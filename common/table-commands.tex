\usepackage{booktabs}
\usepackage{tabularx}
\usepackage{caption}
\renewcommand{\arraystretch}{1.2} %Increase vertical spacing in tables for neatness.

%Declare a caption format that prints only the caption, with no label.
%#1 is the label, #2 is the separator, #3 is the text.
\DeclareCaptionFormat{tablecaptiononly}{#3}
\captionsetup[table]{format=tablecaptiononly}

%\begin{tabularx} cannot be used inside \newenvironment (see section 5 of tabularx manual).
%We would use \tabularx instead.
%However, environments using \tabularx cannot be nested in further environment definitions.
%Without being able to nest things, this would likely lead to code duplication.
%So we will be using commands, not environments, for tables.
%Note that this means we can use the tabularx environment as normal.

\newcommand\simpletableinner[2]{%Arguments: alignment, table
	\rowcolors{2}{white}{gray!20}
	
	%tabularx is necessary in the case of X-type columns.
	%But using tabularx without X-type columns causes the hrules to span wider than the table.
	%So detect whether X-type columns are in use, and use regular tabular if not.
	\IfSubStr{#1}{X}{
		\begin{tabularx}{\linewidth}{#1}
			#2
		\end{tabularx}
	}{
		\begin{tabular}{#1}
			#2
		\end{tabular}
	}
}

\newcommand\simpletable[4]{%Arguments: title, label, alignment, table
	\begin{table}
		\centering
		\caption{#1} %TODO
		\label{table:#2}
		\simpletableinner{#3}{#4}
	\end{table}
}

%For rotated table headers.
%TODO: Deal with the sapcing between this and \toprule.
\newcommand\rothead[1]{\multicolumn{1}{c}{\makebox[1em][l]{\rotatebox{90}{#1}}}}

\newcommand\tableref[1]{\nameref{table:#1}}
