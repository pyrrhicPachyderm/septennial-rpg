\usepackage{booktabs}
\usepackage{tabularx}
\renewcommand{\arraystretch}{1.2} %Increase vertical spacing in tables for neatness.

%\begin{tabularx} cannot be used inside \newenvironment (see section 5 of tabularx manual).
%We would use \tabularx instead.
%However, environments using \tabularx cannot be nested in further environment definitions.
%Without being able to nest things, this would likely lead to code duplication.
%So we will be using commands, not environments, for tables.
%Note that this means we can use the tabularx environment as normal.

\newcommand\simpletableinner[2]{%Arguments: alignment, table
	\rowcolors{2}{white}{gray!20}
	
	%tabularx is necessary in the case of X-type columns.
	%But using tabularx without X-type columns causes the hrules to span wider than the table.
	%So detect whether X-type columns are in use, and use regular tabular if not.
	\IfSubStr{#1}{X}{
		\begin{tabularx}{\linewidth}{#1}
			#2
		\end{tabularx}
	}{
		\begin{tabular}{#1}
			#2
		\end{tabular}
	}
}

\newcommand\simpletable[2]{%Arguments: alignment, table
	\begin{center}
	\simpletableinner{#1}{#2}
	\end{center}
}
